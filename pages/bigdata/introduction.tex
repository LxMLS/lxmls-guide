In this lab, we will work with Amazon.com's Web Services, a cloud based solution
to run some simple analyses. Then, in the next lab, we will build on these
tools to construct a larger learning system.

\section{Introduction}

AWS is a set of services of different types. We will focus on the \emph{Elastic
Compute Cluster}, EC2.

In this lab, we will be logging in to Linux-based machines using a command line
interface. If you are not familiar with the command line, you may not grasp all
the nuances

\section{Running Your First Hadoop Job}

You should have received a piece of paper with the following four pieces of
information:

\begin{enumerate}
\item A host name
\item A password
\item A ``AWS Key''
\item A ``AWS Secret
\end{enumerate}

We will ignore the first two items for the moment.  Your hostname will look
something like \texttt{ec2-54-234-117-69.compute-1.amazonaws.com} and you
should be able to log in with username \texttt{ubuntu} and the password you
were given! This is your personal server.

If you are using the command line, you can just type:

ssh ubuntu@ec2-54-234-117-69.compute-1.amazonaws.com

(accept the RSA key and log in with the password we provided you).

The machine has all of the necessary Python packages for you to work. It also
has a standard collection of editors and other tools. It is an Ubuntu based
installation and you can access super user powers with \texttt{sudo} (Don't
worry if you did not understand this last sentence).


\section{MapReduce \& Hadoop}

The initial idea of MapReduce was an implementation by google, but very soon
Hadoop appeared as an open-source implementation of the same paradigm.

\subsection{Word Count}

Here is the problem:

\begin{enumerate}
\item You have a collection of documents, there are many of them.
\item You want to count how many times each word appears in the whole collection.
\end{enumerate}

Counting words in a collection of documents can be parallelized in the following way:

\begin{enumerate}
\item You process each document by itself, counting how many times each word
appears. This is the \emph{map} step. Notice that each document can be
processed in parallel by different machines. The result for each file is a
dictionary of a count for each word.
\item You sum up the counts for each word to get the final count. This is the
\emph{reduce} step. Notice that you can now parallelize over words.
\end{enumerate}

In this tutorial, we will be using the package \texttt{MrJob} (which was
developed by Yelp, the company behind the epinomious website; but is available
as open source, under the Apache license). On your Amazon machine, the package
is already installed with all its dependencies, so you can just go ahead and
use it.

We are going to implement the word count algorithm now:

\begin{python}
# Import the necessary libraries:
from mrjob.job import MRJob

class WordCount(MRJob):
    def mapper(self, _, doc):
        c = {}
        # Process the document
        for w in doc.split():
            if w in c:
                c[w] += 1
            else:
                c[w] = 1

        # Now, output the results
        for w,c in c.items():
            yield w,c

    def reducer(self, key, cs):
        yield key, sum(cs)

wc = WordCount()
wc.run()
\end{python}

You can save this as \texttt{wordcount.py} and run this on your machine
directly. We will also make use of the \texttt{/datasets} directory that is
preinstalled on your machine:

python wordcount.py /datasets/small-text-file.txt

\subsubsection{Running Word Count using Hadoop}

We are now going to use the AWS Key and AWS Secret. These are like a username
and password for Amazon Webservices.

We can run the exact same code as before using EMR:

\subsubsection{Running Word Count on Wikipedia}

This file is not very big, but running on a cluster of machine allows us to
scale up run on Wikipedia

\subsection{Using Naïve Bayes for Language Detection}

The implementation of binary Naïve Bayes is very similar to counting words: we just
need to keep track of the class of each document and maintain.


