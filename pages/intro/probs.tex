Probability is the mathematical language for quantifying uncertainty. 
The {\bf sample space} $\mathcal{X}$ is the set of possible outcomes of an experiment.
{\bf Events} are subsets of $\mathcal{X}$.

\begin{example}
[{\em discrete space}] Let H denote ``heads'' and T denote ``tails.'' If we toss a coin twice, then $\mathcal{X}=\{ HH, HT, 
TH, TT\}$. The event that the first toss is heads is $A= \left\{HH, HT \right\}$.
\end{example}

Sample space can also be {\em continuous} (eg., $\mathcal{X}= \mathbb{R}$). The union of events $A$ and $B$ is 
defined as $A \bigcup B = \{ \omega \in \mathcal{X}\,\,|\,\, \omega \in A \vee \omega \in B\}$. 
If $A_1$, $\ldots$, $A_n$  is a sequence of sets 
then $\bigcup\limits_{i=1}^{n}A_{i} = \{ \omega \in \mathcal{X}\,\,|\,\, \omega \in A_{i} \text{ for at least one i}\}$.
We say that  $A_1$, $\ldots$, $A_n$ are {\bf disjoint} or {\bf mutually exclusive} if $A_{i} \cap A_{j} = \varnothing$ 
whenever $i \neq j$.

\vspace{0.1in}
We want to assign a real number $P(A)$  to every event $A$, called the {\bf probability} of $A$. We also
call $P$ a {\bf probability distribution} or {\bf probability measure}. 

\bigskip

\fbox
{\begin{minipage}[h]{0.9\linewidth} 
\begin{definition}
A function $P$ that assigns a real number $P(A)$ to each event A is a {\bf probability distribution} or a {\bf probability
measure} if it satisfies the three following axioms: \\

Axiom 1: $P(A) \geq 0$ {\em for every A} \\
Axiom 2: $P(\mathcal{X}) = 1$ \\
Axiom 3: If  $A_1$, $\ldots$, $A_n$ are disjoint then 
\begin{equation*}
P\left(\bigcup\limits_{i=1}^{n} A_{i}\right) = \sum\limits_{i=1}^{n} P(A_{i}).
\end{equation*}
\end{definition}
\end{minipage}}

\vspace{0.1in}
One can derive many properties of $P$ from these axioms:
\begin{eqnarray*}
P(\varnothing) &= &0 \\
A \subseteq B &\Rightarrow &P(A) \leq P(B) \\
0 \leq &P(A) &\leq 1 \\
P(A') &= &1 - P(A) \hspace{0.2in}(\text{A' is the complement of A}) \\
P(A \cup B) &=& P(A) + P(B) - P(A \cap B) \\
A \cap B = \phi &\Rightarrow &P\left( A \cup B\right) = P(A) + P(B).
\end{eqnarray*}

An important case is when events are {\bf independent}, this is also 
a usual approximation which lends several practical advantages to the computation of 
joint probability.\\

\fbox
{\begin{minipage}[h]{0.9\linewidth} 
\begin{definition}

Two events $A$ and $B$ are {\bf independent} if 
\begin{equation}
P(AB) = P(A)P(B)
\end{equation}
often denoted as $A \independent B $. A set of events $\{A_{i}: i \in I\}$ is independent if 
\begin{equation*}
P\left(\bigcap\limits_{i \in J} A_{i}\right) = \prod\limits_{i \in J}P(A_{i})
\end{equation*}
for every finite subset J of I.
\end{definition}
\end{minipage}}

\vspace{0.1in}
For events $A$ and $B$, where $P(B) > 0$, {\bf conditional probability} of $A$ given $B$ has occurred is defined as 

\begin{equation} 
P(A|B)=\frac{P(AB)}{P(B)}.
\label{eq:condprob}
\end{equation}

Events $A$ and $B$ are independent if and only if $P(A|B)=P(A)$. This follows from the definitions
of independence and conditional probability.


A preliminary result that forms the basis for the famous Bayes' theorem is the law of total probability
which states that if $A_{1},\ldots,A_{k}$ is a partition of $\mathcal{X}$, then for any event B,
\begin{equation}
P(B)=\sum_{i=1}^{k}P(B|A_{i})P(A_{i}).
\label{eq:totprob}
\end{equation}


Using Equations \ref{eq:condprob}  and \ref{eq:totprob}, one can derive the famous Bayes' theorem.

\vspace{0.1in}
\fbox
{\begin{minipage}[h]{0.9\linewidth} 
\begin{theorem}
{\bf (Bayes' Theorem)} Let $A_{1}, \ldots A_{k}$ be a partition of $\mathcal{X}$ such that $P(A_{i}) > 0$ for each i. If $P(B) > 0$
then, for each $i=1,\ldots, k,$
\begin{equation}
P(A_{i}|B)=\frac{P(B|A_{i})P(A_{i})}{\sum_{j}P(B|A_{j})P(A_{j})}.
\end{equation}
\end{theorem}
\end{minipage}}

\begin{remark}
$P(A_{i})$ is called the {\bf prior probability} of A and $P(A_{i}|B)$ is the {\bf posterior probability} of A.
\end{remark}

\begin{remark}
In Bayesian Statistical Inference, Bayes' theorem is used to compute the estimates of parameters for distributions 
from data; Where, prior is the initial {\em belief} about the parameters, likelihood is the distribution function  
of the parameter (usually trained from data) and posterior is the {\em updated belief} about the parameters.
\end{remark}

%The theorem also holds when the sample space is continuous.
\subsection{Probability distribution functions}

A {\bf random variable} is a mapping $X:\mathcal{X} \rightarrow \mathbb{R}$ that assigns a real number $X(\omega)$
to each outcome $\omega$. Given a random variable X, an important function called the {\bf cumulative distributive
function} (or {\bf distribution function}) is defined as:

\begin{definition}
The {\bf cumulative distribution function} CDF $F_{X}: \mathbb{R} \rightarrow [0,1]$ of a random variable $X$ is defined by
$F_X(x) = P(X \leq x)$.
\end{definition}

The CDF is important because it captures the complete information about the random variable. The CDF is right-continuous,
non-decreasing and is normalized (lim$_{x\rightarrow -\infty} F(x) = 0$ and lim$_{x\rightarrow \infty} F(x)=1$).


\begin{example} 
[{\em discrete CDF}]  Flip a fair coin twice and let X be the random variable indicating the number of heads. Then 
$P(X=0)=P(X=2)= 1/4$ and $P(X=1)=1/2$. The distribution function is 
\begin{equation*}
F_{X}(x)=\left\{\begin{array}{cc}
0 & x<0\\
1/4 & 0 \leq x < 1 \\
3/4 & 1 \leq x < 2 \\
1 & x \geq 2.
\end{array}\right.
\end{equation*}
\end{example}

\begin{definition}
$X$ is discrete if it takes countable many values $\{x_{1},x_{2},\ldots\}$. We define the {\bf probability function} or
{\bf probability mass function} for $X$ by
\begin{equation*}
	f_{X}(x)= P (X=x).
\end{equation*}
\end{definition}


\fbox
{\begin{minipage}[h]{0.9\linewidth} 
\begin{definition}
A random variable $X$ is {\bf continuous} if there exists a function $f_{X}$ such that $f_{X} \geq 0$ for all $x$,
$\int\limits_{-\infty}^{\infty} f_{X}(x)dx = 1$ and for every $a\leq b$
\begin{equation}
P(a < X < b) = \int\limits_{a}^{b}f_{X}(x)dx.
\end{equation}
The function $f_{X}$ is called the {\bf probability density function} (PDF). We have that
\begin{equation*}
F_{X}(x) = \int\limits_{-\infty}^{x}f_{X}(t)dt
\end{equation*}
and $f_{X}(x)=F_{X}^{'}(x)$ at all points $x$ at which $F_{X}$ is differentiable.
\end{definition}
\end{minipage}}


\vspace{0.1in}
A discussion of a few important distributions and related properties:

%Basic concepts of probability theory... Space of events, probability, conditional probability, etc
%Concept of conjugate prior
%For each distribution talk about 
\subsection{\label{bernoulli} Bernoulli}
The {\bf Bernoulli distribution} is a discrete probability distribution that takes 1 with the success probability $p$ and 0 with the failure probability $q=1-p$.
A single bernoulli trial is parametrized with the success probability $p$, and the input $k\in \{0,1\}$ (1=success, 0=failure), and can be expressed as
\begin{equation*}
\label{bernoulli-eq}
f(k;p) = p^kq^{1-k} = p^k(1-p)^{1-k}
\end{equation*}

\subsection{\label{binomial} Binomial}
The probability distribution for the number of successes in $n$ Bernoulli trials is called a {\bf Binomial distribution}, which is also a discrete distribution.
The Binomial distribution can be expressed as 
exactly $j$ successes is 
\begin{equation*}
f(j,n;p)= \left(\begin{array}{c}
n \\
j \end{array}\right) p^{j}q^{n-j} = \left(\begin{array}{c}
n \\
j \end{array}\right) p^{j}(1-p)^{n-j}
\end{equation*}
where $n$ is the number of Bernoulli trails with probability $p$ of success on each trial.

\subsection{\label{categorical} Categorical}
The Categorical distribution (often conflated with the Multinomial distribution, in fields such as Natural Language Processing), is another generalization of the Bernoulli distribution, allowing the definition of a set of possible outcomes, rather than simply the events "success" and "failure" defined in the Bernoulli distribution.
Considering a set of outcomes indexed from $1$ to $n$, the distribution takes the form of
\begin{equation*}
f(x_i;p_1,...,p_n) = p_i.
\end{equation*}
Where parameters $p_1,...,p_n$ is the set with the occurrence probability of each outcome. Note that we must ensure that $\sum_{i=1}^np_i=1$, so we can set $p_n = \sum_{i=1}^{n-1}p_i=1$.

\subsection{\label{multinomial} Multinomial}

The Multinomial distribution is a generalization of the Binomial distribution and the Categorical distribution, since it considers multiple outcomes, as the Categorial distribution, and multiple trials, as in the Binomial distribution.
Considering a set of outcomes indexed from $1$ to $n$, the vector $x_1,...,x_n$, where $x_i$ indicates the number of times the event with index $i$ occurs, follows the Multinomial distribution
\begin{equation*}
f(x_1,...,x_n;p_1,...,p_n) = \frac{n!}{x_{1}!...x_{n}!} p_{1}^{x_{1}}...p_{n}^{x_{n}}.
\end{equation*}
Where parameters $p_1,...,p_n$ is the occurrence probability of the respective outcome. 


%\begin{theorem}{\bf (Multinomial theorem)} Let $m$ and $n$ be positive integers. Let $A$ be the set of vectors $x=(x_{1},\ldots,x_{n})$ 
%such that each $x_{i}$ is a non-negative integer and $\sum_{i=1}^{n}x_{i}=m$. Then for any real 
%numbers $p_{1},\ldots,p_{n}$,
%\begin{equation*}
%(p_{1}+\ldots+p_{n})^{m} = \sum\limits_{x\in A} \frac{m!}{x_{1}!\ldotp\ldots\ldotp x_{n}!} p_{1}^{x_{1}}\ldotp\ldots\ldotp p_{n}^{x_{n}}.
%\end{equation*}
%\end{theorem}

\subsection{\label{gaussian} Gaussian Distribution}
A very important theorem in probability theory called the {\bf Central Limit Theorem}
states that, under very general conditions, if we sum a very large number of mutually independent
random variables, then the distribution of the sum can be closely approximated by a certain specific
continuous density called the normal (or Gaussian) density. The normal density function with parameters
$\mu$ and $\sigma$ is defined as follows:
\begin{equation*}
f_{X}(x)=\frac{1}{\sqrt{2\pi}\sigma}e^{-(x-\mu)^2/2\sigma^2}, \text{   }-\infty < x < \infty.
\end{equation*}


\begin{figure}[h]
\begin{center}
    \includegraphics[width=0.6\columnwidth]{figs/intro/normal.png}
  \caption{\label{fig:normaldist} Normal density for two sets of parameter values.}
\end{center}
\end{figure}

Fig. \ref{fig:normaldist}, compares a plot of normal density for the cases $\mu=0$ and $\sigma=1$, and 
$\mu=0$ and  $\sigma=2$.

\subsection{Conjugate Priors}
%\fbox
%{\begin{minipage}[h]{0.9\linewidth} 
\begin{definition}
let $\mathcal{F}= \{f_{X}(x|s), s \in \mathcal{X}\}$ be a class of likelihood functions; let $\mathcal{P}$ be a class
of probability (density or mass) functions; if, for any $x$, any $p_{S}(s) \in \mathcal{P}$, and any $f_{X}(x|s) \in \mathcal{F}$,
the resulting a posteriori probability function $p_{S}(s|x) = f_{X}(x|s)p_{S}(s)$ is still in $\mathcal{P}$, then $\mathcal{P}$ is 
called a conjugate family, or a family of {\bf conjugate priors}, for $\mathcal{F}$.
\end{definition}
%\end{minipage}}
%\gka{An example here for conjugate families}

\subsection{Maximum Likelihood Estimation}
Until now, we assumed that, for every distribution, the parameters $\theta$ are known and are used when we calculate $p(x|\theta)$.
There are some cases where the values of the parameters are easy to infer, such as the probability $p$ getting a head using a fair coin, used on a Bernoulli or Binomial distribution.
However, in many problems, these values are complex to define, and it is more viable to estimate the parameters using the data $x$. 
For instance, in the example above with the coin toss, if the coin is somehow tampered to have a biased behavior, 
rather than examining the dynamics of the structure of the coin to infer a parameter for $p$, 
a person could simply throw the coin $n$ times and count the number of heads $h$ and set $p=\frac{h}{n}$. In doing this, the person is using the data $x$ to estimate $\theta$.

With this in mind, we will now generalize this process. We define the probability $p(\theta|x)$, which is probability of the parameter $\theta$, given the data $x$. 
This probability is called {\bf likelihood} $\likelihood(\theta|x)$ and measures how well the parameter $\theta$ models the data $x$. This can be defined in terms of the distribution $f$ as
\begin{equation*}
\likelihood(\theta|x_1,...,x_n)=\prod_{i=1}^n f(x_i|\theta)
\end{equation*}
where $x_1,...,x_n$ are iid samples.

To understand this concept better, we go back to the tampered coin example again. Suppose that we throw the coin 5 times and get the sequence [1,1,1,1,1] (1=head, 0=tail). 
Using the Bernoulli distribution~\ref{bernoulli-eq} $f$ to model this problem, we get the following likelihood values:
\begin{itemize}
\item $\likelihood(0,x) = f(1,0)^5 = 0^5 = 0$
\item $\likelihood(0.2,x) = f(1,0.2)^5 = 0.2^5 = 0.00032$
\item $\likelihood(0.4,x) = f(1,0.4)^5 = 0.4^5 = 0.01024$
\item $\likelihood(0.6,x) = f(1,0.6)^5 = 0.6^5 = 0.07776$
\item $\likelihood(0.8,x) = f(1,0.8)^5 = 0.8^5 = 0.32768$
\item $\likelihood(1,x) = f(1,1)^5 = 1^5 = 1$
\end{itemize}

If we get the sequence [1,0,1,1,0] instead, the likelihood values would be:
\begin{itemize}
\item $\likelihood(0,x) = f(1,0)^3f(0,0)^2 = 0^3\times 1^2 = 0$
\item $\likelihood(0.2,x) = f(1,0.2)^3f(0,0.2)^2 = 0.2^3\times 0.8^2 = 0.00512$
\item $\likelihood(0.4,x) = f(1,0.4)^3f(0,0.4)^2 = 0.4^3\times 0.6^2 = 0.02304$
\item $\likelihood(0.6,x) = f(1,0.6)^3f(0,0.6)^2 = 0.6^3\times 0.4^2 = 0.03456$
\item $\likelihood(0.8,x) = f(1,0.8)^3f(0,0.8)^2 = 0.8^3\times 0.2^2 = 0.02048$
\item $\likelihood(1,x) = f(1,1)^5 = 1^3\times 0^2 = 0$
\end{itemize}

We can see that the likelihood is highest when the distribution $f$ with parameter $p$ is the best fit for the observed samples.
Thus, the best estimate for $p$ according to $x$ would the value where $\likelihood(p,x)$ is highest. 

The value of the parameter $\theta$ with the highest likelihood is called {\bf maximum likelihood estimate(MLE)} and is defined as
\begin{equation*}
\hat{\theta}_{mle}=argmax_{\theta}\likelihood(\theta|x)
\end{equation*}

Finding this for our example is relatively easy, since we can simply derivate the likelihood function to find the absolute maximum. 
For the sequence [1,0,1,1,0], the likelihood would be given as
\begin{equation*}
\likelihood(p,x) = f(1,p)^3f(0,p)^2 = p^3(1-p)^2
\end{equation*}

And the MLE estimate would be given by:
\begin{equation*}
\frac{\delta \likelihood(p,x)}{\delta p}=0
\end{equation*}
which resolves into
\begin{equation*}
p_{mle}=0.6
\end{equation*}

\begin{exercise}
Over the next couple of exercises we will make use of the Galton dataset, a
dataset of heights of fathers and sons from the 1877 paper that first discussed
the ``regression to the mean'' phenomenon. This dataset has 928 pairs of numbers.

\begin{itemize}
\item Use the \texttt{load()} function in the \texttt{galton.py} file to load
the dataset. The file is located under the \texttt{lxmls/readers} folder. Type the following in your Python interpreter:
\begin{verbatim}
import galton as galton
GaltonData = galton.load()
\end{verbatim}
\item What are the mean height and standard deviation of all the people in the
sample? What is the mean height of the fathers and of the sons?
\item Plot a histogram of all the heights (you might want to use the
\texttt{plt.hist} function and the \texttt{ravel} method on arrays).
\item Plot the height of the father versus the height of the son.
\item You should notice that there are several points that are exactly the same
(e.g., there are 21 pairs with the values 68.5 and 70.2). Use the \texttt{?}
command in ipython to read the documentation for the \texttt{numpy.random.randn}
function and add random jitter (i.e., move the point a little bit) to the
points before displaying them. Does your impression of the data change?
\end{itemize}
\end{exercise}
