\subsection{Desktops vs. Laptops}

If you have decided to use one of our provided desktops, all installation procedures have been carried out. You merely need to go to the \verb+lxmls-toolkit-student+ folder inside your home directory and start working! You may go directly to section \ref{sec:SolvingExercises}. If you wish to use your own laptop, you will need to install Python, the required Python libraries and download the LXMLS code base. It is important that you do this as soon as possible (before the school starts) to avoid unnecessary delays. Please follow the install instructions. 

\subsection{Downloading the labs version from GitHub student branch}

The code of LxMLS is available online at GitHub. There are two branches of the code: the \verb+master+ branch contains fully functional code. \textbf{important}: The \verb+student+ branch contains the same code with some parts deleted, which you must complete in the following exercises. Download the \verb+student+ code by going to

\begin{verbatim}
https://github.com/LxMLS/lxmls-toolkit
\end{verbatim}

\noindent and select the \verb+student+ branch in the dropdown menu. This will reload the page to the corresponding branch. Now you just need to click the \verb+clone or download + button to obtain the lab tools in a zip format:

\begin{verbatim}
lxmls-toolkit-student.zip
\end{verbatim}

After this you can unzip the file where you want to work and enter the unzipped folder. This will be the place where you will work. 

\subsection{Installing Python from Scratch with Anaconda}

If you are new to Python the best option right now is the Anaconda platform. You can find installers for Windows, Linux and OSX platforms here

\begin{verbatim}
https://www.anaconda.com/download/
https://anaconda.org/pytorch/pytorch
\end{verbatim}

Finally install the LXMLS toolkit symbolically. This will allow you to modify the code and see the changes take place immediately.

\begin{verbatim}
python setup.py develop
\end{verbatim}

\noindent The guide supports both Python2 and Python3. We strongly recommend that you use Python3 as Python2 is being deprecated. 

\begin{verbatim}
\end{verbatim}

\subsection{Installing with Pip}

If you are familiar with Python you will probably be used to the pip package installer. In this case it might be more easy for you to install the packages yourself using pip and a virtual environment. This will avoid conflicting with existing python installations. To install and create a virtual environment do

\begin{verbatim}
cd lxmls-toolkit-student
sudo pip install virtualenv
virtualenv venv 
source venv/bin/activate
pip install --editable .
\end{verbatim}

If you also need to install a new python version, e.g. your systems Python is still Python2 and you can not change this, you can virtualize different Python versions as well. Have a look at pyenv. This will hijack your python binary and allow you
switch between Python 2, 3 or install concrete versions for a particular folder. To install pyenv

\begin{verbatim}
git clone https://github.com/pyenv/pyenv.git ~/.pyenv
\end{verbatim}

\noindent and add the following to your .bashrc or .bash\_profile

\begin{verbatim}
export PYENV_ROOT="$HOME/.pyenv"
export PATH="$PYENV_ROOT/bin:$PATH"
eval "$(pyenv init -)
\end{verbatim}

be careful if you already redefine Python's path in these files. If you do not feel comfortable with this its better to ask for help. Once installed you can do the following in the LxMLS folder

\begin{verbatim}
source ~/.bashrc
pyenv install 3.6.0 
pyenv local 3.6.0
\end{verbatim}

\noindent to install and set the version of python for that folder to Python 3.6.0


\subsection{(Advanced Users) Forking and cloning the code on GitHub}

It might be the case that you feel very comfortable with scientific Python, know some git/GitHub and want to extend/improve our code base. In that case you can directly clone the project with

\begin{verbatim}
git clone https://github.com/LxMLS/lxmls-toolkit.git 
cd lxmls-toolkit/
git checkout student
pip install --editable .
\end{verbatim}

\textbf{Note:}
\noindent If you are experiencing issues on Windows, you would probably have to install pytorch first.
\begin{verbatim} 
pip install https://download.pytorch.org/whl/cpu/torch-1.1.0-cp36-cp36m-win_amd64.whl
pip install https://download.pytorch.org/whl/cpu/torchvision-0.3.0-cp36-cp36m-win_amd64.whl
\end{verbatim}

\noindent If you want to contribute to the code-base, you can make pull requests to the \textit{develop} branch or raise issues.

\subsection{Deciding on the IDE and interactive shell to use}

An Integrated Development Environment (IDE) includes a text editor and various tools to debug and interpret complex code. 

\textbf{Important:} As the labs progress you will need an IDE, or at least a good editor and knowledge of pdb/ipdb. This will not be obvious the first days since we will be seeing simpler examples.

Easy IDEs to work with Python are PyCharm and Visual Studio Code, but feel free to use the software you feel more comfortable with. PyCharm and other well known IDEs like Spyder are provided with the Anaconda installation.

Aside of an IDE, you will need an interactive command line to run commands. This is very useful to explore variables and functions and quickly debug the exercises. For the most complex exercises you will still need an IDE to modify particular segments of the provided code. As interactive command line we recommend the Jupyter notebook. This also comes installed with Anaconda and is part of the pip-installed packages. The Jupyter notebook is described in the next section. In case you run into problems or you feel uncomfortable with the Jupyter notebook you can use the simpler iPython command line.

\subsection{Jupyter Notebook}

Jupyter is a good choice for writing Python code. It is an interactive computational environment for data science and scientific computing, where you can combine code execution, rich text, mathematics, plots and rich media. The Jupyter Notebook is a web application that allows you to create and share documents, which contains live code, equations, visualizations and explanatory text. It is very popular in the areas of data cleaning and transformation, numerical simulation, statistical modeling, machine learning and so on. It supports more than 40 programming languages, including all those popular ones used in Data Science such as Python, R, and Scala. It can also produce many different types of output such as images, videos, LaTex and JavaScript. More over with its interactive widgets, you can manipulate and visualize data in real time.

\noindent The main features and advantages using the Jupyter Notebook are the
following:

\begin{itemize}

\item In-browser editing for code, with automatic syntax highlighting, indentation, and tab completion/introspection.

\item The ability to execute code from the browser, with the results of computations attached to the code which generated them.

\item Displaying the result of computation using rich media representations, such as HTML, LaTeX, PNG, SVG, etc. For example, publication-quality figures rendered by the matplotlib library, can be included inline.

\item In-browser editing for rich text using the Markdown markup language, which can provide commentary for the code, is not limited to plain text.

\item The ability to easily include mathematical notation within markdown cells using LaTeX, and rendered natively by MathJax.

\end{itemize}

\noindent The basic commands you should know are

\begin{table}[!h]
\begin{center}
\begin{tabular}{|l|l|}
\hline
Esc              & Enter command mode\\
Enter            & Enter edit mode\\
\hline
up/down          & Change between cells\\
Ctrl + Enter     & Runs code on selected cell\\
Shift + Enter    & Runs code on selected cell, jumps to next cell\\
\hline
restart button   & Deletes all variables (useful for troubleshooting)\\ 
\hline
\end{tabular}
\end{center}
\caption{\label{tb::jupyterbasiccommands}Basic Jupyter commands}
\end{table}

\noindent A more detailed user guide can be found here:

\begin{verbatim}
http://jupyter-notebook-beginner-guide.readthedocs.io/en/latest/index.html
\end{verbatim}
