\section{Today's assignment}
Today's class will demystify the Transformer architecture \cite{vaswani2017attention}. Transformers have become the dominant model in Natural Language Processing pushing performance of all tasks through models like BERT \cite{devlin2018bert} and GPT \cite{brown2020language}. In this class we will review the architecture of Transformers. The focus of the class is to understand and implement the self-attention mechanism in pytorch. Fro this we will use an annotated version of Karpathy's miniGPT library.


\section{Transformer Architectures}

The basic building block of the Transformer combines a large Feed forward network with a multi-head self-attention layers. We have seen feed forwards in previous days and we will focus here in multi-head self-attention. Refer to previous days for details on the basics of feed-forwards.

Transformer blocks can be found in three types: Encoder, Decoder and Encoder-Decoder 

\begin{enumerate}
\item Encoders: map a sequence of $T$ observations, e.g. some word or sub-word units $x_1 \cdots x_T$ to a hidden representation of size $H$, yielding a matrix of embeddings of size $(H, T)$. These contextual embeddings can be used to build other models on top by adding extra layers as e.g. BERT.
\item Decoders: given a sequence of $t$ observations, e.g. some word or sub-word units $x_{<t} = x_1 \cdots x_{t-1}$ provide a hidden representation for element $x_t$. This can be used to predict the next word given some partial sentence e.g. as in GPT. 
\item Cross-Attention-Decoders: map a sequence $x$ to another of different size $y$. For this they first Encode $x$ using an encoder. And then use a modified Decoder, that uses an additional attention mechanism read the Encoder values to generate embeddings for $y_t$ given $y_{<t}$ and the $\mathrm{Encoder}(x)$. The original Transformer used for machine translation is the best example.
\end{enumerate}

Though with very different roles the three architectures are very close to each other and imply only minor modifications. We will focus here on the Encoder architecture first, and then we will explain the other two.

\section{Encoder Architecture}

\subsection{Simplified Encoder Architecture}

the encoder architecture can be succinctly described as stacking a number of $N$ blocks on top of each other combining a feed forward $\mathrm{FF}$ and Multi-Head Attention $\mathrm{MHA}$ sub-blocks. A single block is defined as

\begin{equation}
e^{n+1} = \mathrm{FF}(\mathrm(MHA(e^n)).
\end{equation}

The input to the first block $e^0$ is the sum of position $P$ and non-contextual embeddings $E$ of the input. Assuming $x_1 \cdots x_T$ is a sequence of integers (indices to a vocabulary of $V$ symbols) we have that

\begin{equation}
e^{0}_t = \mathrm{E} \cdot \mathrm{1}_{x_t} + \mathrm{P} \cdot \mathrm{1}_t \quad \mbox{for} \quad t=1 \cdots T
\end{equation}

where $\mathrm{1}_{x_t}$ and $\mathrm{1}_t$ are indicator, i.e. one-hot, vectors for the token content (vocabulary symbol) and the token position. $\mathrm{E} \in \mathbb{R}^{H \times V}$ is the non-contextual embedding matrix for each symbol in the vocabulary and $P^{H \times \tau}$ is the position embedding matrix, where  $\tau-1$ is the furthest position supported.\\

\noindent The $\mathrm{FF}$ and $\mathrm{MHA}$ are given by

\begin{equation}
FF(z) = W^2\cdot \mathrm{gelu}(W^1 \cdot z)
\end{equation}

\noindent with weight matrices $W^2 \in \mathbb{R}^{H \times 4H}$ and $W^1 \in \mathbb{R}^{4H \times H}$, that expand and contract the hidden dimension $H$, and

\begin{equation}
MHA(z) =  W^o \cdot
\begin{bmatrix}
    W^V_1 \cdot z \cdot \underset{i \rightarrow}{\mathrm{softmax}}\left( \frac{1}{A} \left(W^K_1 z\right)^T W^Q_1 z \right)\\
    W^V_2 \cdot z \cdot \underset{i \rightarrow}{\mathrm{softmax}}\left( \frac{1}{A} \left(W^K_2 z\right)^T W^Q_2 z \right)\\
    \cdots\\
    W^V_H \cdot z \cdot \underset{i \rightarrow}{\mathrm{softmax}}\left( \frac{1}{A} \left(W^K_D z\right)^T W^Q_H z \right)\\
\end{bmatrix}\nonumber
\end{equation}

where we have $D$ attention heads. Each head contracts the hidden dimension $W^K, W^Q, W^V \in \mathbb{R}^{H / D \times H}$ into a space of size $H / D$ (this has practical implementation consequences). Outputs of all heads are concatenated and projected again with $W^o \in \mathbb{R}^{H \times H}$.

\subsection{Adding Dropout, Layer Normalization and Residuals}

For simplicity, we have left a number of additional operations that can be added as a wrapper arround $\mathrm{FF}$ and $\mathrm{MHA}$ operations. These can be expressed as

\begin{equation}
C(f)(x) = x + \mathrm{dropout}(f(\mathrm{layernorm}(x)))
\end{equation}

where $f = \{\mathrm{FF}, \mathrm{MHA}\}$.


\section{Decoder Architecture}

This is identical to the Encoder architecture with two differences

\begin{enumerate}
\item We feed if a partial sequence $x_{<t}$ and take the last output $h_{t-1}$ as the hidden vector for $x_t$
\item We mask every head of $\mathrm{MHA}$ to prevent any value of time $p$ to depend on values of $>p$
\end{enumerate}

\section{Encoder-Decoder Architecture}

We need two modifications

\begin{enumerate}
\item Encode the input with an encoder $e^N = \mathrm{Encoder}(x)$
\item Modify self-attention to use a second query input to be projected with $W^Q$ while the other two project $E^N$ $d^{m+1} = \mathrm{FF}(\mathrm(CMHA(e^N, MHA(d^m))).$
\end{enumerate}

